\documentclass[ignorenonframetext,]{beamer}
\setbeamertemplate{caption}[numbered]
\setbeamertemplate{caption label separator}{: }
\setbeamercolor{caption name}{fg=normal text.fg}
\beamertemplatenavigationsymbolsempty
\usepackage{lmodern}
\usepackage{amssymb,amsmath}
\usepackage{ifxetex,ifluatex}
\usepackage{fixltx2e} % provides \textsubscript
\ifnum 0\ifxetex 1\fi\ifluatex 1\fi=0 % if pdftex
  \usepackage[T1]{fontenc}
  \usepackage[utf8]{inputenc}
\else % if luatex or xelatex
  \ifxetex
    \usepackage{mathspec}
  \else
    \usepackage{fontspec}
  \fi
  \defaultfontfeatures{Ligatures=TeX,Scale=MatchLowercase}
\fi
% use upquote if available, for straight quotes in verbatim environments
\IfFileExists{upquote.sty}{\usepackage{upquote}}{}
% use microtype if available
\IfFileExists{microtype.sty}{%
\usepackage{microtype}
\UseMicrotypeSet[protrusion]{basicmath} % disable protrusion for tt fonts
}{}
\newif\ifbibliography
\hypersetup{
            pdftitle={Income and Cash Flow Statments},
            pdfborder={0 0 0},
            breaklinks=true}
\urlstyle{same}  % don't use monospace font for urls
\usepackage{longtable,booktabs}
\usepackage{caption}
% These lines are needed to make table captions work with longtable:
\makeatletter
\def\fnum@table{\tablename~\thetable}
\makeatother

% Prevent slide breaks in the middle of a paragraph:
\widowpenalties 1 10000
\raggedbottom

\AtBeginPart{
  \let\insertpartnumber\relax
  \let\partname\relax
  \frame{\partpage}
}
\AtBeginSection{
  \ifbibliography
  \else
    \let\insertsectionnumber\relax
    \let\sectionname\relax
    \frame{\sectionpage}
  \fi
}
\AtBeginSubsection{
  \let\insertsubsectionnumber\relax
  \let\subsectionname\relax
  \frame{\subsectionpage}
}

\setlength{\parindent}{0pt}
\setlength{\parskip}{6pt plus 2pt minus 1pt}
\setlength{\emergencystretch}{3em}  % prevent overfull lines
\providecommand{\tightlist}{%
  \setlength{\itemsep}{0pt}\setlength{\parskip}{0pt}}
\setcounter{secnumdepth}{0}

\title{Income and Cash Flow Statments}
\date{}

\begin{document}
\frame{\titlepage}

\begin{frame}{Goals}

\begin{itemize}
\tightlist
\item
  Make what you learned about taxes and depreciation understandable to
  others.
\item
  Use accounting conventions to convey your cash flow projections.
\end{itemize}

\end{frame}

\begin{frame}{Background}

\begin{itemize}
\tightlist
\item
  We will be focusing on the cash flow effects of some investment.
\item
  We will not be looking at the whole business, just the thing that
  changed.
\item
  We are looking at a counter-factual, ``How would cash flow change?''
\end{itemize}

\end{frame}

\begin{frame}{The Example Counter-factual}

\begin{itemize}
\tightlist
\item
  How would cash flow change if we purchased a new inventory management
  system that reduces our need for inventory?
\item
  How would cash flow change if we purchased a new delivery truck with a
  loan?\\
\item
  Is cash flow better than without the loan?
\end{itemize}

\end{frame}

\begin{frame}{Sign Conventions: Revenue}

\begin{table}[h]
\begin{tabular}{l||c|c||p{3cm}}
\multicolumn{4}{c}{Income Statement}\\
  &Year 0  &Year 1  &Note\\
\hline
        Operating Revenue   &   100 & (500) &\\
        Operating Expenses  &   &   &\\
        Interest    &   &   &\\
        Depreciation    &   &   &\\
        \hline
        Taxable Income  &   &   &\\
        Tax &   &   &\\
        \hline
        Net Income  &   &   &\\
\end{tabular}
\end{table}

More revenue than current plan in year zero and less revenue than
current plan in year one.

\end{frame}

\begin{frame}{Sign Conventions: Expenses}

\begin{table}[h]
\begin{tabular}{l||c|c||p{3cm}}
\multicolumn{4}{c}{Income Statement}\\
  &Year 0  &Year 1  &Note\\
\hline
        Operating Revenue   &   100 & (500) &\\
        Operating Expenses  &   50& (700)   &\\
        Interest    &   &   &\\
        Depreciation    &   &   &\\
        \hline
        Taxable Income  &   &   &\\
        Tax &   &   &\\
        \hline
        Net Income  &   &   &\\
\end{tabular}
\end{table}

More expenses than current plan in year zero and fewer expenses than
current plan in year one.

\end{frame}

\begin{frame}{Sign Conventions: Interest}

\begin{table}[h]
\begin{tabular}{l||c|c||p{3cm}}
\multicolumn{4}{c}{Income Statement}\\
  &Year 0  &Year 1  &Note\\
\hline
        Operating Revenue   &   100 & (500) &\\
        Operating Expenses  &   50& (700)   &\\
        Interest    &   &   300&\\
        Depreciation    &   &   &\\
        \hline
        Taxable Income  &   &   &\\
        Tax &   &   &\\
        \hline
        Net Income  &   &   &\\
\end{tabular}
\end{table}

Increase in interest expense, relative to current plan, in year 1. Note
that this is just the interest expense, not a full loan payment.

\end{frame}

\begin{frame}{Sign Conventions: Depreciation}

\begin{table}[h]
\begin{tabular}{l||c|c||p{3cm}}
\multicolumn{4}{c}{Income Statement}\\
  &Year 0  &Year 1  &Note\\
\hline
        Operating Revenue   &   100 & (500) &\\
        Operating Expenses  &   50& (700)   &\\
        Interest    &   &   300&\\
        Depreciation    &   4000&   1600&\\
        \hline
        Taxable Income  &   &   &\\
        Tax &   &   &\\
        \hline
        Net Income  &   &   &\\
\end{tabular}
\end{table}

If you replaced an asset with one that cost less, and had less
depreciation in years zero and one, then you would see negative values.
Note that we have half the usual depreciation because we are selling the
asset in year 1.

\end{frame}

\begin{frame}{Sign Conventions: Taxable Income}

Subtract your expenses from your revenue.

\begin{table}[h]
\begin{tabular}{l||c|c||p{3cm}}
\multicolumn{4}{c}{Income Statement}\\
  &Year 0  &Year 1  &Note\\
\hline
        Operating Revenue   &   100 & (500) &\\
        Operating Expenses  &   50& (700)   &\\
        Interest    &   &   300&\\
        Depreciation    &   4000&   1600&\\
        \hline
        Taxable Income  &   (3950)& (1700)  &\\
        Tax &   &   &\\
        \hline
        Net Income  &   &   &\\
\end{tabular}
\end{table}

Year 0 Taxable Income = 100 -50 - 4000

Year 1 Taxable Income = -500 - (-700) -300 - 3200/2

\end{frame}

\begin{frame}{Sign Conventions: Tax}

Apply your tax rate (40\%) to the Taxable Income. Note that increases in
taxes are positive but decreases in taxes are shown as a negative.

\begin{table}[h]
\begin{tabular}{l||c|c||p{3cm}}
\multicolumn{4}{c}{Income Statement}\\
  &Year 0  &Year 1  &Note\\
\hline
        Operating Revenue   &   100 & (500) &\\
        Operating Expenses  &   50& (700)   &\\
        Interest    &   &   300&\\
        Depreciation    &   4000&   1600&\\
        \hline
        Taxable Income  &   (3950)& (1700)  &\\
                Tax &   (1580)& (680)&\\
        \hline
        Net Income  &   &   &\\
\end{tabular}
\end{table}

This says that your taxes will be \emph{lower} because of the action you
are taking.

\end{frame}

\begin{frame}{Sign Conventions: Net Income}

Subtract Tax from Taxable Income to get Net Income, i.e., profits.

\begin{table}[h]
\begin{tabular}{l||c|c||p{3cm}}
\multicolumn{4}{c}{Income Statement}\\
  &Year 0  &Year 1  &Note\\
\hline
        Operating Revenue   &   100 & (500) &\\
        Operating Expenses  &   50& (700)   &\\
        Interest    &   &   300&\\
        Depreciation    &   4000&   1600&\\
        \hline
        Taxable Income  &   (3950)& (1700)  &\\
                Tax &   (1580)& (680)&\\
        \hline
        Net Income &    (2370)& (1020)&\\
\end{tabular}
\end{table}

This says that our profits are lower than they would be otherwise.

\end{frame}

\begin{frame}{Cash Flow}

Cash from operations is just copied from the income statement

\begin{table}
\begin{tabular}{l||c|c||p{3cm}}
        \multicolumn{4}{c}{Cash Flow Statement}\\
  &Year 0  &Year 1  &Note\\        
    Operations& &   &\\
    \hspace{.25in}Net Income    &   (2370)& (1020)&\\
    \hspace{.25in}Depreciation  &   4000&   1600&\\
    Investments &   &   &\\
            &   &   &\\
\hspace{.25in}Working Capital   &   &   &\\
        \hspace{.25in}Gains Tax &   &   &\\
            &   &   &\\
        
    Finance &   &   &\\
            &   &   &\\
        \hline
        Net Cash Flow   &   &   &\\
\end{tabular}
\end{table}

Keep in mind that the convention is cash inflows are positive and cash
outflows are negative.

\end{frame}

\begin{frame}{Sign Conventions: Asset Purchase}

The Asset is a 20K, 5-Year Asset

\begin{table}
\begin{tabular}{l||c|c||p{3cm}}
        \multicolumn{4}{c}{Cash Flow Statement}\\
  &Year 0  &Year 1  &Note\\        
    Operations& &   &\\
    \hspace{.25in}Net Income    &   (2370)& (1020)&\\
    \hspace{.25in}Depreciation  &   4000&   1600&\\
    Investments &   &   &\\
\hspace{.25in}The Asset         &(20000)    &   &\\
\hspace{.25in}Working Capital   &   &   &\\
        \hspace{.25in}Gains Tax &   &   &\\
            &   &   &\\
        
    Finance &   &   &\\
            &   &   &\\
        \hline
        Net Cash Flow   &   &   &\\
\end{tabular}
\end{table}

Outflows of cash are negative. Note depreciation reflects sale in year
1.

\end{frame}

\begin{frame}{Sign Conventions: Asset Sale}

Sell the asset in year 1 for 19000

\begin{table}
\begin{tabular}{l||c|c||p{3cm}}
        \multicolumn{4}{c}{Cash Flow Statement}\\
  &Year 0  &Year 1  &Note\\        
    Operations& &   &\\
    \hspace{.25in}Net Income    &   (2370)& (1020)&\\
    \hspace{.25in}Depreciation  &   4000&   1600&\\
    Investments &   &   &\\
\hspace{.25in}The Asset         &(20000)    &   19000&\\
\hspace{.25in}Working Capital   &   &   &\\
        \hspace{.25in}Gains Tax &   &   &\\
            &   &   &\\
        
    Finance &   &   &\\
            &   &   &\\
        \hline
        Net Cash Flow   &   &   &\\
\end{tabular}
\end{table}

Inflows of cash are positive.

\end{frame}

\begin{frame}{Sign Conventions: Asset Sale}

Gains tax has opposite sign convention. Positive means pay.

\begin{table}
\begin{tabular}{l||c|c||p{3cm}}
        \multicolumn{4}{c}{Cash Flow Statement}\\
  &Year 0  &Year 1  &Note\\        
    Operations& &   &\\
    \hspace{.25in}Net Income    &   (2370)& (1020)&\\
    \hspace{.25in}Depreciation  &   4000&   1600&\\
    Investments &   &   &\\
\hspace{.25in}The Asset         &(20000)    &   19000&\\
\hspace{.25in}Working Capital   &   &   &\\
        \hspace{.25in}Gains Tax &   &   &\\
            &   &   &\\
        
    Finance &   &   &\\
            &   &   &\\
        \hline
        Net Cash Flow   &   &   &\\
\end{tabular}
\end{table}

Book Value = 20000 - 4000 - 3200/2 = 14400

\end{frame}

\begin{frame}{Sign Conventions: Gains Tax}

Gains tax has opposite sign convention. Positive means pay.

\begin{table}
\begin{tabular}{l||c|c||p{3cm}}
        \multicolumn{4}{c}{Cash Flow Statement}\\
  &Year 0  &Year 1  &Note\\        
    Operations& &   &\\
    \hspace{.25in}Net Income    &   (2370)& (1020)&\\
    \hspace{.25in}Depreciation  &   4000&   1600&\\
    Investments &   &   &\\
\hspace{.25in}The Asset         &(20000)    &   19000&\\
\hspace{.25in}Working Capital   &   &   &\\
        \hspace{.25in}Gains Tax &   &1840   &\\
            &   &   &\\
        
    Finance &   &   &\\
            &   &   &\\
        \hline
        Net Cash Flow   &   &   &\\
\end{tabular}
\end{table}

\(Gains~Tax = (Salvage - Book~Value) Tax~Rate =\)

\end{frame}

\begin{frame}{Sign Conventions: Net Cash Flow}

Add up cash but remember gains tax must be subtracted.

\begin{table}
\begin{tabular}{l||c|c||p{3cm}}
        \multicolumn{4}{c}{Cash Flow Statement}\\
  &Year 0  &Year 1  &Note\\        
    Operations& &   &\\
    \hspace{.25in}Net Income    &   (2370)& (1020)&\\
    \hspace{.25in}Depreciation  &   4000&   1600&\\
    Investments &   &   &\\
\hspace{.25in}The Asset         &(20000)    &   19000&\\
\hspace{.25in}Working Capital   &   &   &\\
        \hspace{.25in}Gains Tax &   &1840   &\\
            &   &   &\\
        
    Finance &   &   &\\
            &   &   &\\
        \hline
        Net Cash Flow   &   (18370)& 1.324\times 10^{4} &\\
\end{tabular}
\end{table}

\end{frame}

\begin{frame}{Two More Conventions}

We still need to show how to treat loans and working capital.

\begin{itemize}
\tightlist
\item
  Loans

  \begin{itemize}
  \tightlist
  \item
    Funding recorded on cash flow
  \item
    Payments are split into interest expense (Income Statement) and
    principal payment (Cash Flow Statement)
  \end{itemize}
\item
  Working Capital Changes

  \begin{itemize}
  \tightlist
  \item
    Changes in inventory level and changes in average accounts
    receivable.
  \item
    Record how the level changes from period to period.
  \end{itemize}
\end{itemize}

\end{frame}

\begin{frame}{From the Loans Slides}

This is the amortization table for a \$10,000, 10\% per year loan with
three annual payments.

\begin{longtable}[]{@{}lllll@{}}
\toprule
Payment Number & Payment & Interest & Principal & Balance\tabularnewline
\midrule
\endhead
0 & & & & 10,000\tabularnewline
1 & 4,021 & 1,000 & 3,021 & 6,979\tabularnewline
2 & 4,021 & 698 & 3,323 & 3,656\tabularnewline
3 & 4,021 & 366 & 3,656 & 1.2e-11\tabularnewline
\bottomrule
\end{longtable}

\begin{itemize}
\tightlist
\item
  Payments are given by \(10000 (A|P, i = .1, 3) = 4,021.15\)
\item
  Interest expense is the balance remaining times the effective interest
  rate per payment period.
\item
  Principal payment is the payment less the interest expense.
\item
  Balance remaining is the previous balance remaining less then
  principal payment.
\end{itemize}

\end{frame}

\begin{frame}{From the Loans Slides}

\begin{longtable}[]{@{}lllll@{}}
\toprule
Payment Number & Payment & Interest & Principal & Balance\tabularnewline
\midrule
\endhead
0 & & & & 10,000\tabularnewline
1 & 4,021 & 1,000 & 3,021 & 6,979\tabularnewline
2 & 4,021 & 698 & 3,323 & 3,656\tabularnewline
3 & 4,021 & 366 & 3,656 & 1.2e-11\tabularnewline
\bottomrule
\end{longtable}

\begin{itemize}
\tightlist
\item
  Loan funding shows in the cash flow statement under finance as a
  positive number.
\item
  Interest expenses are shown on the income statement as a positive
  number.
\item
  Principal payments are show on the cash flow statement as a negative
  number.
\end{itemize}

\end{frame}

\begin{frame}{In the Income and Cash Flow Statement}

\begin{table}[h]
\begin{small}
\begin{tabular}{l||c|c|c|c|}
\multicolumn{5}{c}{Income Statement}\\
  &Year 0  &Year 1  & Year 2 & Year 3 \\
\hline
        Operating Revenue   & &     & & \\
        Operating Expenses  &   &   & & \\
        Interest    &   &   1000&698 &366 \\
        Depreciation    &   &   & & \\
        \hline
        Taxable Income  &0  &(1000) &(698) &(366) \\
        Tax &   0 & (400)   &(279.2) & (146.4) \\
        \hline
        Net Income  & 0 &(600)  & (418.8)& (219.6)\\

\end{tabular}

\begin{tabular}{l||c|c|c|c|}
        \multicolumn{5}{c}{Cash Flow Statement}\\
  &Year 0  &Year 1  & Year 2 & Year 3 \\
\hline
    Operations& &   & &\\
    \hspace{.25in}Net Income    & 0 & (600) &(418.8) &(219.6)\\
    \hspace{.25in}Depreciation  &   &   & &\\
    Investments &   &   & &\\
    Finance &   &   & &\\
    \hspace{.25in}      Loan    &   10000 & & &\\
    \hspace{.25in}      Loan    Payments&   &(3021) &(3323) &(3656)\\
        \hline
        Net Cash Flow   & 10000 & (3621)    & (3741.8)& (3875.6)\\
\end{tabular}

\end{small}
\end{table}

\end{frame}

\begin{frame}{Working Capital Changes}

These are common:

\begin{itemize}
\tightlist
\item
  Increases in sales necessitates a larger inventory (Working Capital)
\item
  More sales mean greater average invoices outstanding.
\end{itemize}

The logic is tricky only in the sense that you are showing the changes
in these levels as cash flows.

\end{frame}

\begin{frame}{Example Inventory Changes}

This shows what happens with average inventory changes from
year-to-year.

\begin{longtable}[]{@{}llllll@{}}
\toprule
Year & 1 & 2 & 3 & 4 & 5\tabularnewline
\midrule
\endhead
Inventory & 1000 & 1200 & 900 & 900 & 800\tabularnewline
Working Capital & 0 & -200 & 300 & 0 & 100\tabularnewline
\bottomrule
\end{longtable}

\begin{itemize}
\tightlist
\item
  Increases in inventory require more cash, outflows.
\item
  Decreases free up cash, inflows.
\item
  Gets complicated when the current plan has inventory levels changing.
  We will not do this (Difference in the differences).
\end{itemize}

\end{frame}

\begin{frame}{Summary}

\begin{itemize}
\tightlist
\item
  This is just a way of presenting the things that you have already done
  using accounting conventions.
\item
  Understandable by everyone that knows the conventions.
\item
  Only a few transactions that I mix and match:

  \begin{itemize}
  \tightlist
  \item
    Increase/Decrease in revenue
  \item
    Increase/Decrease in sales
  \item
    Asset purchase/sale with gains tax
  \item
    Loan and loan repayment.
  \item
    Working Capital Changes.
  \end{itemize}
\item
  Worked Examples
  \url{http://ec314-pdx-edu.wikidot.com/q4:income-and-cash-flow-statements}
\end{itemize}

\end{frame}

\end{document}
