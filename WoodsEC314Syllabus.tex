\documentclass[letterpaper,10pt]{article}
\usepackage{fullpage}
\usepackage{hyperref}
\usepackage{draftwatermark}
\SetWatermarkScale{4}
\newcommand{\Term}{Fall 2014}
\newcommand{\Office}{on Monday from 10:30-11:30}
\newcommand{\Piazza}{\url{http://piazza.com/pdx/fall2014/ec314/home}}

\date{}
%opening
\title{Public and Private Investment Analysis\\ 
EC 314\\
\Term}

\author{James Woods}

\begin{document}

\maketitle

\section{Introduction}

This course will examine the tools required to analyze expenditures
that yield benefits over time -- investments. The use of accounting
documents and a focus on the time value of money will allow students
to analyze choices in a variety of security, loan, and equipment
investment decisions.

The main objectives are to prepare engineers for their PE exams and to
provide engineers, economists and others with a shared language and
understanding of basic finance topics.

This is not an online course, but it is expected that you have access to a
computer to complete the homework and access class resources.  

\input{Hybrid}

\section{Prerequisites}

There are no prerequisites for this course. However, a reasonable math
background, MTH 95, is a practical requirement. The math placement assessment for MTH 111, testing concepts from MTH 70 and 95, i.e., high school algebra, will be given on the first day.  If you don't do well, you
should reconsider enrolling in this course.

\section{Contact Information}

If each of you sent me an email, text, or message, once a week, I would be unable to respond to all your questions.  Rather than emailing questions, I encourage you to post your questions on Piazza. If you have any problems or feedback for the developers, email team@piazza.com.

You can find our class page at: \Piazza.  There is also a link in D2L.

Please get signed up as soon as possible. I have set up a
folder for each of the class modules. If you want to discuss the class material, need a clarification or have a concern about a quiz question, look to see if someone has asked the same question and if it has been answered. 

If you have a question about course mechanics, interpretation of the
syllabus, or where something is located, there is a folder for
that called ``Logistics''.

I will check piazza on a regular basis during regular business hours, and answer questions. I also expect you and your peers to help to each other and will indicate the good answers and make clarifications.  

If you send me an email with a question that should be asked in a piazza, I
will ask you to post it there rather than answer in email.

The TAs assigned to the class is Peter Hulseman. 

Peter Hulseman's contact information is:
\begin{itemize}
\item Office Hours: W 1:30-3:30 and Thurs 10-1, CH 230
\item Email/gtalk: hulseman@pdx.edu
\end{itemize}

 
My office is in CH 241-O.  The best ways of contacting me, in
decreasing order of effectiveness:
\begin{itemize}
\item Googletalk/hangout: woodsj@pdx.edu
\item Phone/text: (503) 465-4883
\item Email:woodsj@pdx.edu (Answered only on M W and F)
\item D2L email: Do not use.  I will not respond.
\end{itemize}

You are encouraged to use the IM function of your PSU email account rather than email. This will allow me to get back to you more quickly and more conversationally. Please be aware that I will likely not reply till the next day if you IM me after 6pm or on the weekend.  

My office hours will be held \Office ~through the last week of class, not including finals week. There is no need to make an appointment for these hours -- just come.

If you can't attend regular office hours, please check my calendar link on D2L. I will make a limited number of 15 minute slots available each week. \emph{If you make an appointment and fail to show up without first canceling, I will penalize you one quiz grade.}  

% These slots are intended for you to ask questions about course material and not satisfy the oral exam requirements described below.  There is a separate link for oral exams.


\section{Textbooks and Other Sources}
I have assigned Chan S. Park ``Contemporary Engineering Economics
4ed''.  If the 4th edition is unavailable, the 3rd or 5th editions are also
acceptable but \emph{all readings and suggested questions will be from
  the 4th edition.} I am pretty comfortable with any engineering
economics textbook you choose, but it is your responsibility to figure
out the relevant readings from the descriptions of the subject matter.
The table of contents for the 4th edition (\url{http://web.pdx.edu/~woodsj/Teaching/table\%20of\%20contents.pdf}) is posted to help you
make the correspondences.  Many students have found the textbook unnecessary given the online resources.

``Schaum's Outline of Engineering Economics'' by Sepulveda, Souder and Gottfried is a useful second reference if you are having trouble with time value of money.

\subsection{Online Resources}
We will make significant use of online resources and technology.  The
three essential tools are:
\begin{itemize}
\item D2L
\item Course Wiki
\item Piazza
% \item Google+ Hangouts
\end{itemize}

D2L is our main platform for homework, i.e., online quizzes.  Should you have technical problems with D2L
or your computer please resolve them yourself with the general campus
help desk or D2L help (\url{http://www.psuonline.pdx.edu/}).  You will
also find a tutorial on how to use D2L on your D2L landing page.
\emph{You are responsible for learning how to use D2L.}


The course D2L landing page will have several important links:
\begin{itemize}
\item EC314 Wiki, where most of the supplementary material, videos and
  podcasts reside.
\item Schedule an Appointment, where you can arrange times to meet
  other than office hours for help or an oral exam.
\item Calendar of Topics and Deadlines, for deadlines and due dates.
\item Table of Contents for 4th Edition Book
\end{itemize}

Pay particular attention to the course content tab in the upper
left. This is where most of the online work starts.  Each module will
be structured with one or more of the following:


\begin{itemize}
  % \item Goal: What we are trying to make sure you can do.
\item Preparation: Pages out of the book and wiki you should read.
  \emph{Wiki links are required.}  Former students have suggested that these should read and watched before attempting the quizzes.  Don't start the quizzes and then use the book and wiki as a reference.
  
%   \item Practice Assignment: This is an online assignment that you
%     may take as many times as you wish and does not count toward
%     your grade.
\item A `Pre' assignment: This counts towards your grade.
  You may make make up to two attempts of the `Pre' assignments before you can attempt the `Post' assignment.  
  Your score is equal
  to the average of the two attempts. Note that each time you take a quiz the questions and answers may change.

\item A `Post' assignment: This is an online assignment that counts
  towards your grade and is equal to two `Pre' assignments.  You will
  only be allowed to take these quizzes once.  Please note that Post quizzes may contain questions from previous modules.

  The pre and post quizzes are timed so please do not start them and
  then close them.  Only start the quizzes when you have the time to
  finish them.  Please note that I dislike it when students take
  quizzes at the same time from the same IP address.

\end{itemize}


On the first day of class we will vote on when the online assignments will be due.  

The pre and post quizzes are timed so please do not start them and
then close them.  Only start the quizzes when you have the time to
finish them.   
% Students have suggested that starting quizzes so that they overlap office hours or a tutorial is very useful.


The calendar tool will show Quiz Deadlines. 
% You have a lot of latitude on when you
%   finish topics. Statistically, there is an inverse relationship between when students start on the material and the grade they receive.  The later you start, the lower your grade. All modules should be completed
%   by the last week of class -- not the last day of finals week.  
  
These are hard deadlines.  Having a disruption in D2L or internet connectivity in the hour or two before does not change the deadline, it means you should plan on finishing the quizzes in advance of the deadline.

Piazza will be used for news and other announcements and has the advantage of being available independently of D2L and the PSU login mechanism.  

If you have questions related to course administration, post them in
the ``logistics'' folder by
composing a message with a brief version of your question in the
subject line. This way we can form a knowledge base that everyone can
use.  If you email me a question like this, I will just ask you to
post the question in this folder.  Save yourself a step and start with a post.

% \subsubsection{Tutorials}
%  I will offer up to three tutorial sections weekly.
%   Times for the next week will be determined by vote. A 
%   poll will be released on Monday and close
%   Thursday at 5pm. I will choose \emph{up to} three one-hour periods with the
%   highest number of votes for the next week's tutorials.  Additional times can be proposed in follow up forum posts.  Please note that my schedule is the primary limitation.


\subsubsection{EC314 Wiki}
The second most important resource is the wiki.  Most of the
supplementary reading material, videos, screencasts, and podcasts are
collected here. You should visit the ``Course Outline'' link on the
left \url{http://ec314-pdx-edu.wikidot.com/outline} to see what we will be
doing.

A link to the wiki may be found on the D2L landing page.  You must create both a wikidot account and ask for access to the course wiki.  Create the account first, log in, and ask for access by filling out the request form.  I log in once a day and allow everyone access.  An account is required so that it is easy for you to make corrections to the wiki.

This resource is outside of PSU and the help desk will not provide
assistance.  If you have questions please post with a  
``logistics'' tag or look directly to wikidot
for help.  You and I are both customers. I have no special status.

If you intend on editing pages or making contributions please look at
the ``how to'' pages and try some things out in the sandbox.

\subsection{Netiquette}
Since we will be having some online interaction, we should be clear
about what we should and should not do.

In class I can see what is going on at all times.  In the online
environment, I can't see everything and I can't react as quickly to
what I do see.  Students are expected to be particularly trustworthy
both in terms of their own conduct, and in reporting inappropriate
conduct to me via email.

I will not tolerate profanity, obscenity, or remarks intended to
discount or dismiss either a person or his or her point of view. I
will not count such messages for grade. They will be deleted from the
message board and returned to you for revision, failure to do so will
result in loss of points.

If your inappropriate message is in the public forum, I will warn you
once in that public space. If you continue with inappropriate conduct
in the public spaces, your access to them will be denied for the rest
of the class with the resultant loss of points. If your inappropriate
conduct is in private, a student code of conduct violation complaint
may be filed.

\subsection{Attendance and Online Participation}

If you choose to attend class, please prepare by
completing the required readings and reviewing your notes from the
previous lecture. If you choose not to attend a class, you are still
responsible for acquiring notes, handouts, and any announced schedule
changes from other students.  You will be ineligible for pop quizzes
or the demonstrations of basic skills.

Don't let your children be a barrier to attending class. Bringing your
children to class occasionally is tolerated and encouraged.

Online participation is not optional.  That is how we do homework.

\section{Assessments and Grade Policy}

Your grade in the class will be based on your performance on the
online assignments.  There are no midterms and there is
not a traditional final exam, but there will be a final exam period.

Every lecture has a chance of one or more pop quizzes. Completing the
in-class quizzes, regardless of performance, results in full credit.

There will be practice quizzes and two types of graded quizzes, `Pre' and `Post', on D2L.  A `Pre'
quiz with 20 points is worth the same as a `Pre' quiz of 3 points as
far as your grade goes.  Points are only comparable within a quiz, not
between quizzes.  `Post' quizzes are worth twice as much as `Pre' quizzes.
% 
% The optimal study pattern is to:
% \begin{itemize}
% \item Read the required material.
% \item Try some of the practice quiz questions.
% \item Wait a day.
% \item Review the material again and do more of the practice questions.
% \item If confident, take the `Pre' quiz, reviewing concepts between each of the two attempts.
% \item Wait a day.
% \item Review and take the `Post' quiz.
% \end{itemize}


Grades will be based on your class rank, subject to the minimum
requirements described below. At the end of the term I will create a
weighted average score and rank the students -- dropping the two
lowest pre quiz scores or lowest post quiz. The dividing lines between letter grades will be
drawn such that no student is near a dividing line. In this way no
student will ever be, ``just one point from an A.'' You are in a very
real competition for grades in this class. If someone cheats it harms
your chances of getting an A. 

I typically end up with \emph{about} 30\%
A's and 30\% B's in my in-person and hybrid courses.  The distribution is usually lower in online courses because of a larger number of students starting assignments at the last minute.   

The two dropped scores are intended to be a buffer for technical
failures, illness, bad days and other bad luck.  I will take care of
this adjustment at the end of term.  You need not do anything.


There are additional requirements for students wishing to be eligible for a grade higher than a C.  Note that these are not required if you are taking the class P/NP and will have no effect on your grade.


\subsection{Minimum Requirements for Grades Higher than C}

The course is focused on four essential skills. You must
demonstrate proficiency in all four to receive a course grade higher than a C:

\begin{itemize}

\item Accounting: Knowledge of the definitions and locations of a
  small subset of items found on income statements, balance sheets and
  cash flow statements.

\item Time Value: The ability to represent arbitrary cash flows
  consisting of common patterns, such as constant series and linear
  gradient series, in factor notation and compute the present worth of
  these series.

\item Choice: The ability choose acceptable and optimal investments,
  given cash flows and an interest rate, using the present worth,
  annual worth and internal rate of return criteria.

\item Income/Cash Flow: The ability to construct simple income/cash
  flow statement for an asset purchase or expense including
  depreciation, interest, taxes, gains tax and working capital to
  support an investment decision based on the present worth, annual
  worth or internal rate of return criteria.
\end{itemize}

Please note that demonstrating one of these skills does not get you a
C, it just means that it will be assigned a grade higher than a C if
your rank in the course warrants it, i.e., a necessary but not sufficient condition.

The final 10 to 15 minutes of each session is reserved to allow
students to demonstrate these skills with a brief written test.
Certifying questions will be made available soon after the topics are introduced in class. Students may attempt one question per class.  All questions
will be scored on a pass/no pass basis and may be returned with the
scoring rubric.  

Students may schedule a 15 minute appointment with the professor, through the links
in D2L. Times will be available after the first two weeks
of the course, but not during final exam week.  Times may be limited
and is the responsibility of the student to plan appropriately.
\emph{Please expect to show your student ID.}

You will also be able to make three attempts in the
final exam period.   To be clear, there is not a traditional final exam.


At the final exam period please bring a calculator, a writing implement, and
your student ID.  You will have your choice of any three essential
question, even three of the same.  They will be graded on a simple
pass/fail scale.  This is intended to be your last chance to
demonstrate proficiency -- not the main chance.  We will not admit you
without your student ID.

Examples of past questions and keys to the grading rubric may be found
in outline on the course wiki under ``Essential Skills Hints ''.

% \subsection{Extra Credit}
% Extra Credit, equivalent to three quiz grades is available for the
% course.  The extra credit quizzes will replace your lowest quiz
% scores.

% Extra credit is earned by making significant contributions to the
% course wiki.  While adding to the wiki without extra credit reward
% is encouraged, extra credit will not be given if the following
% procedure is not followed.

% \begin{itemize}
% \item Check the list of to do items on the To Do page
%   \url{http://ec314-pdx-edu.wikidot.com/to-do}.  There is also a
%   link in the left panel.
% \item Send me an email telling me which item you intend to complete
%   before you start work. Your email must include a deadline. I will
%   change the wiki to show that you are working on the item and the
%   deadline.
% \item I will not accept ex-post notification.
% \item Once you have completed the project, you will send me an email
% \item I may ask for changes or improvements before the extra-credit
%   is granted.
% \item They must be completed by November 27th.

% \end{itemize}
% \subsection{Additional Requirements for Grades Higher than B}
% 
% This is a requirement in addition to the one above, for
% students that would like to be considered for a grade higher than a
% B. Completing this assignment does not ensure an A, it just makes it
% possible should your rank in the class warrant the grade.
% 
% This assignment comes in two parts. After you complete the 'Income/Cash Flow' module construct a reasonably complete proposal for a choice problem that includes:
% 
% \begin{itemize}
% 
% \item The context of the choice, for example, considering two different business vehicles but one has a higher resale value.
% \item Assumptions about:
% \begin{itemize}
% \item Revenue and cost changes;
% \item Cost basis and resale of any assets;
% \item Any loans;
% \item Changes in working capital requirements
% \end{itemize}
% \item The main problems you will have to deal with, e.g., a gains tax or a loan with many payments per year.
% \end{itemize}
% 
% This will be submitted to the ``Income Cash Flow Proposal'' dropbox folder in D2L by  March 3rd for approval.  If not approved, any revisions to the proposal must be completed by March 13th.  In general only one revision will be allowed before your proposal is rejected.  Please note that D2L will not allow you to submit a proposal until you have completed the Post Income and Cashflow quiz.
% 
% A spreadsheet showing the Income and Cashflow statement as well as narrative, not to exceed one page, explaining the statement and your investment choice must be submitted to the ``Income Cash Flow Final'' dropbox by March 20th.  This will be evaluated on a pass/no pass basis. If your submission is made prior to March 13th, you may receive feedback and an opportunity for resubmission.
% 


\subsection{Other Rules}
\begin{itemize}


\item Begging for grades will result in an immediate lowering of your
  course grade by a full letter grade.

\item Go to office hours at the first sign of trouble -- not as a last
  resort.

\item In this classroom, we support and value diversity.  To do so requires that we:
\begin{itemize}
   \item Respect the dignity and essential worth of all individuals
   \item Promote a culture of respect toward all individuals
    \item Respect the privacy, property, and freedom of others
    \item Reject bigotry, discrimination, violence, or intimidation of any kind
    \item Practice personal and academic integrity and expect it from others
   \item Promote the diversity of opinions, ideas, and backgrounds, which is
    the lifeblood of a university
\end{itemize}

 For additional information, please see the Office of Affirmative Action \& Equal Opportunity at \url{http://www.pdx.edu/diversity/affirmative-action}.


\item Accommodations are collaborative efforts between students, faculty, and the Disability Resource Center.  If you have a documented disability and require accommodation, you must arrange to meet with the course instructor prior to or within the first week of the term.  The documentation of your disability must come in writing from the Disability Resource Center (Faculty letter).  Students who believe they are eligible for accommodations but who have not yet obtained approval through the DRC should contact the DRC immediately.  Reasonable and appropriate accommodations will be provided for students with documented disabilities.  For more information on the Disability Resource Center, please see \url{http://www.drc.pdx.edu/}. 

\item Academic honesty is expected and required of students enrolled
  in this course.  Suspected academic dishonesty in this course will
  be handled according to the procedures set out in the Student Code
  of Conduct.

\item I am sympathetic to family emergencies but you must inform me as
  soon as possible. If the notice is verbal, please email me with your
  understanding of our agreement. All agreements have to be in
  writing.

\item I expect online quizzes to be completed by you without consulting your classmates or other human resources.
\end{itemize}


\subsection{Expected Outline}
A link to a full, detailed, and continuously updated schedule will be
posted on D2L and the course wiki.


\end{document}
