\documentclass[letterpaper,10pt]{article}
\usepackage{fullpage}
\usepackage{hyperref}
% \usepackage{draftwatermark}
% \SetWatermarkScale{4}

\newif\ifhybrid
\newif\ifonline
\newif\ifinperson

% \hybridtrue
% \onlinetrue
 \inpersontrue

\newcommand{\Term}{Spring 2016}
\newcommand{\Office}{on Monday from 3-4:30}
\newcommand{\Piazza}{\url{ http://piazza.com/pdx/spring2016/ec314/home}}
\newcommand{\FinalExam}{at 12:30pm  on June 9th}
% \newcommand{\MidtermExam}{on April 27th}
\newcommand{\DRCDate}{May 2nd }

\date{}
%opening
\title{Public and Private Investment Analysis\\  
EC 314\\
\Term}

\author{James Woods}

\begin{document}
  
\maketitle

\section{Introduction}

This course will examine the tools required to analyze expenditures
that yield benefits over time -- investments. The use of accounting
documents and a focus on the time value of money will allow students
to analyze choices in a variety of security, loan, and equipment
investment decisions. The main objectives are to prepare engineers for their PE and FE exams and to
provide engineers, economists and others with a shared language and
understanding of basic finance topics.


\ifhybrid This is a hybrid class. That means that we only have half the usual amount of classroom time but will have a lot more work on-line. \fi
\ifonline This is a online course and it is expected that you have access to a
computer with sufficient software and hardware to use the the hangout
feature. Please drop this course if you don't have access to a computer with a webcam, microphone and a reasonably fast internet connection.  \fi

\emph{This class has an in-person final exam at \FinalExam. If you are taking the exam at the PSU testing center, you must schedule your exam to start at the same time as the rest of the class. It is recomended that you schedule appointments with the testing center as soon as possible to avoid taking the exam with the rest of the class.}

\section{Prerequisites}

There are no prerequisites for this course. However, a reasonable math
background, MTH 95, is a practical requirement. 

\ifonline
  
  \else
    The math placement assessment for MTH 111, testing concepts from MTH 70 and 95, i.e., high school algebra, will be given on the first day.  If you don't do well, you should reconsider enrolling in this course.
\fi

\section{Contact Information}

If each of you sent me an email, text, or message, once a week, I would be unable to respond to all your questions.  Rather than emailing questions, I encourage you to post your questions on Piazza (\Piazza). If you have any problems or feedback for the developers, email team@piazza.com. Please get signed up as soon as possible. 

There is a folder for each of the class modules. If you want to discuss the class material, need a clarification or have a concern about a quiz question, look to see if someone has asked the same question and if it has been answered.  If you have a question about course mechanics, interpretation of the syllabus, or where something is located, there is a folder for that called ``Logistics''. I will check piazza on a regular basis during regular business hours, and answer questions. I expect you and your peers to help to each other; I will indicate the good answers and make clarifications.  

If you send me an email with a question that should be asked in a piazza, I
will ask you to post it there rather than answer in email.

If your question is not appropriate for piazza, the best ways of contacting me, in
decreasing order of effectiveness:
\begin{itemize}
\item IM/hangout: woodsj@pdx.edu
\item Phone/text: (503) 465-4883
\item Email:woodsj@pdx.edu (Answered only on M W and F)
\item D2L email: Do not use.  I will not respond.
\end{itemize}

You are encouraged to use the IM function of your PSU email account, or text, rather than email. This will allow me to get back to you more quickly and more conversationally. Please be aware that I will likely not reply till the next day if you IM me after 5pm or on the weekend.  

My office is in CH 241-O.  \emph{I will hold office hours \Office ~through the last week of class, not including finals week.} There is no need to make an appointment for these hours -- just come.  D2L has a link that allows you to attend office hours remotely via hangout.

If you can’t attend regular office hours, please check my calendar \url{https://woodsj.youcanbook.me/}. I will make a limited number of 15 minute slots available each week. If you make an appointment and fail to show up without first canceling, I will penalize you one pre quiz grade.  

The TAs assigned to the class is Peter Hulseman. 

Peter Hulseman's contact information is:
\begin{itemize}
\item Office Hours: W 1:30-3:30 and Thurs 10-1, CH 230
\item Email/gtalk: hulseman@pdx.edu
\end{itemize}

 
\section{Textbooks and Other Sources}

Most enginering economics texts have just about the same material, with some small variations and additions. I have assigned Chan S. Park ``Contemporary Engineering Economics
4ed'' as a convenience to students that have their textbooks paid for by a third party. It contains enough of the acounting material that is normally not included in an engineering economics course, but is included in this one to make it part of a universities studies cluster course. 


I am pretty comfortable with any engineering economics textbook you choose, but it is your responsibility to figure
out the relevant readings from the descriptions of the subject matter.
The table of contents for the 4th edition (\url{http://web.pdx.edu/~woodsj/Teaching/table\%20of\%20contents.pdf}) is posted to help you
make the correspondences.  Many students have found the textbook unnecessary given the online resources.  The goal is to have your textbook cost less than \$30.

``Schaum's Outline of Engineering Economics'' by Sepulveda, Souder and Gottfried is a useful second reference if you are having trouble with time value of money.

\subsection{Online Resources}
We will make significant use of online resources and technology.  The
three main tools are:
\begin{itemize}
\item D2L
\item Course Wiki
\item Piazza
% \item Google+ Hangouts
\end{itemize}

D2L is our main platform for homework, i.e., online quizzes.  Should you have technical problems with D2L
or your computer please resolve them yourself with the general campus
help desk or D2L help (\url{http://www.pdx.edu/oit/d2l}).  You will
also find a tutorial on how to use D2L on your D2L landing page.
\emph{You are responsible for learning how to use D2L.}


The course D2L landing page will have several important links:
\begin{itemize}
  \item EC314 Wiki, where most of the supplementary material, videos and
    podcasts reside.
  \ifonline
    \item Schedule an Appointment, where you can arrange times to meet
    other than office hours for help.
    \else
    \item Schedule an Appointment, where you can arrange times to meet
    other than office hours for help or an oral exam.  
  \fi
  
  \item A hangout link so you can attend office hours via webcam.
  \item Calendar of Topics and Deadlines, for deadlines and due dates.
  \item Table of Contents for 4th Edition Book
\end{itemize}

Pay particular attention to the course content tab in the upper
left. This is where most of the online work starts.  Each module will
be structured with one or more of the following:

\begin{itemize}

\item Preparation: Pages out of the book and wiki you should read. Former students have suggested that these should read and watched before attempting the quizzes.  Don't start the quizzes and then use the book and wiki as a reference.
  
  \item Practice Assignment: This is an online assignment that you may take as many times as you wish and does not count toward your grade. Each of these questions will have a wiki page associated with them. 

\item `Pre' assignments: These count towards your grade.
  Your score is equal to the maximum of up to two attempts. Note that the second time you take the quiz that the questions will be similar but not identical. You must get at least one question right to take the quiz a second time.

\item `Post' assignment: This assignment counts
  towards your grade and is equal to two `Pre' assignments.  You will
  only be allowed to take these quizzes once.  Please note that Post quizzes may contain questions from previous modules and will have open response questions that will be hand graded after the due date.

 
\end{itemize}

 The pre and post quizzes are timed. You have 120 minutes to complete each quiz but the usual time to complete a quiz is approximately 45 minutes.  Please do not start them and
  then close them without submitting answers; you will receive a zero on those quizzes.  

The calendar tool will show quiz deadlines. These are hard deadlines.  Having a disruption in D2L or internet connectivity in the hour or two before the deadline does not change the deadline, it means you should plan on finishing the quizzes in advance of the deadline.

\subsubsection{Piazza}
Piazza will be used for news and other announcements and has the advantage of being available independently of D2L and the PSU login mechanism.  

If you have questions related to course administration, post them in
the ``logistics'' folder by composing a message with a brief version of your question in the subject line. This way we can form a knowledge base that everyone can use.  If you email me a question like this, I will just ask you to post the question in this folder.  Save yourself a step and start with a post.

Many of the D2L quiz questions don't have very informative keys. This is partially because many of the questions have several mathematically correct answers or ask for examples, which are by their nature, unique for each student. \emph{After the due date, feel free to post full questions in piazza.  I and other students will work out an answer key.}


\subsubsection{EC314 Wiki}
The second most important resource is the wiki.  Most of the
supplementary reading material, videos, screencasts, and podcasts are
collected here. You should visit the ``Course Outline'' link on the
left \url{http://ec314-pdx-edu.wikidot.com/outline} to see what we will be
doing.

A link to the wiki may be found on the D2L landing page.  You must create both a wikidot account and ask for access to the course wiki.  Create the account first, log in, and ask for access by filling out the request form.  I log in once a day and allow everyone access.  

An account is required so that it is easy for you to make corrections to the wiki.  If you see something that needs fixing -- fix it.  If it is wrong, we can roll it back.  Don't be shy about making edits.  There are ``how to'' pages and sandbox to help you out.

This resource is outside of PSU and the help desk will not provide
assistance.  You and I are both customers. I have no special status.

\subsection{Netiquette}
Since we will be having some online interaction, we should be clear
about what we should and should not do.

In class, I can see what is going on at all times.  In the online
environment, I can't see everything and I can't react as quickly to
what I do see.  Students are expected to be particularly trustworthy
both in terms of their own conduct, and in reporting inappropriate
conduct to me.

I will not tolerate profanity, obscenity, or remarks intended to
discount or dismiss either a person or his or her point of view. If your inappropriate message is in the public forum, I will warn you
once in that public space. If you continue with inappropriate conduct
in the public spaces, your access to them will be denied for the rest
of the term. If your inappropriate
conduct is in private, a student code of conduct violation complaint
may be filed.

\ifonline

\else
   \subsection{Attendance and Online Participation}
   
   If you choose to attend class, please prepare by completing the required readings and reviewing your notes from the previous lecture. If you choose not to attend a class, you are still responsible for acquiring notes, handouts, and any announced schedule changes from other students. You will be ineligible for demonstrations of basic skills, i.e., the essential questions. 
   
   Don’t let your children be a barrier to attending class. Bringing your children to class occasionally is tolerated and encouraged.
  
  Online participation is not optional. That is how we do homework.

\fi

\section{Assessments and Grade Policy}

\ifonline
  Your grade in the class will be determined by your performance on the online assignments in D2L (80\%) and the in-person final exam (20\%).  Grades will be based on your class rank.

  Only about 30\% of students enrolled at the start of term complete the online version of this course.  The typically grade distribution is often worse than the my in-person and hybrid courses where \emph{about} 30\% of students earn A's and another 30\% earn B's.

\else
  Your grade in the class will be determined by your performance on the online assignments in D2L (70\%) and the final exam (30\%).  Grades will be based on your class rank and limited by your performance on the in-class assessments of essential skills which are detailed in section \ref{sec:essentialSkills}. 
  
  I typically end up with \emph{about} 30\%
  A's and 30\% B's in my in-person and hybrid courses.    
\fi

\subsection{On-line Quizzes}

There will be two types of graded quizzes, `Pre' and `Post', on D2L.  A `Pre'
quiz with 20 points is worth the same as a `Pre' quiz of 3 points as
far as your grade goes.  You will be given the maximum score of the two nearly identical attempts. Points are only comparable within a quiz, not between quizzes.  `Post' quizzes are worth twice as much as `Pre' quizzes.

\ifonline

  The optimal study pattern is to:
  \begin{itemize}
  \item Read the required material.
  \item Try some of the practice quiz questions.
  \item Wait a day.
  \item Review the material again and do more of the practice questions.
  \item If confident, take the `Pre' quiz, reviewing concepts between each of the two attempts. Ideally you should put a day for review of the questions between the two attempts.
  \item Wait a day.
  \item Review and take the `Post' quiz.
  \end{itemize}

\fi


Your quiz score is determined by the sum of the percent correct of the pre quizzes plus twice the sum of the percent correct on each of the post quizzes. I drop either the two lowest pre quiz scores or the lowest post quiz, whichever results in the highest score. The dropped scores are intended to be a buffer for technical
failures, illness, bad days and other bad luck.  I will take care of this adjustment at the end of term.  

\subsection{Final Exam}

The function of the final exams is to provide me with added assurance that online work was completed by the person enrolled in the class.  Because of this, picture ID will be required to enter the the examination room.  \emph{Not taking the final exam will result in a failing grade.}

It has been my experience that students I have suspected of not doing their own work online do so poorly on the exams that they fail the class.

The exams will consist of written, not multiple choice questions, covering the same material as the online quizzes.  Expect problem solving, rather than definitional, questions. I will provide you with a sheet of accounting ratios, tables for depreciation, and formulas for some time value of money calculations.  These materials will be posted on D2L in the course materials section.  In the final exam, you will be allowed a calculator and writing implement.  Phones, smart watches, books, computers are not allowed.

\ifonline

\else
   \subsection{ Necessary but not Sufficient Requirements for Passing Grades}\label{sec:essentialSkills}
  
   The course is focused on four essential skills:
   
   \begin{itemize}
   
   \item Accounting: Knowledge of the definitions and locations of a
     small subset of items found on income statements, balance sheets and
     cash flow statements.
   
   \item Time Value: The ability to represent arbitrary cash flows
     consisting of common patterns, such as constant series and linear
     gradient series, in factor notation and compute the present worth of
     these series.
   
   \item Choice: The ability choose acceptable and optimal investments,
     given cash flows and an interest rate, using the present worth,
     annual worth and internal rate of return criteria.
   
   \item Income/Cash Flow: The ability to construct simple income/cash
     flow statement, e.g., for an asset purchase or expense, including
     depreciation, interest, taxes, gains tax and working capital, to
     support an investment decision based on the present worth, annual
     worth or internal rate of return criteria.
   \end{itemize}
   
  
  You must demonstrate proficiency in all four to receive an A in the course, three of the four to receive a B in the course, and two of the four to receive a C-.  These are requirements in addition to having sufficiently high quiz scores.  In other words, these are necessary but not sufficient requirements for the indicated letter grades.
  
  There are three kinds of opportunities to pass these essential skills:
  \begin{itemize}
      \item In class
      \item Oral exam by appointment
      \item As addendum to the final exam
  \end{itemize}
  
   
  \subsubsection{In Class Essential Questions}
  
  Ten to fifteen minutes of each session is reserved to allow students to demonstrate these skills with a brief written test. Certifying questions will be made available soon after the topics are introduced in class. Students may attempt one question per class and we will bring new questions every week.  All questions will be scored on a pass/no pass basis. 
  
  Your graded essential questions will be available at the will call window in the Economics office, CH 241.  This return procedure is FERPA compliant and I apologize for the inconvenience.
  
  Examples of past questions and keys to the grading rubric may be found
  in outline on the course wiki under ``Essential Skills Hints ''.  Note that the formats of some of the questions have changed over time.
  
  \subsubsection{Essential Questions by Oral Exam}
  
  Students may schedule a 15 minute appointment with the professor through the appointment link. There are often no free appointment slots available in the final two weeks of class and it is the responsibility of the student to plan appropriately. \emph{If you don’t see an available appointment slot, there are no available slots.} There will be some difference in the format of the oral exams relative to the written version. After failed oral exam I will give feedback on how to study and do better the next time. Please expect to show your student ID.
  
  Please refrain from scheduling multiple oral exams on the same day.  If you make more than one appointment, I will cancel all but the earliest.
  
  \subsubsection{ Essential Questions as Addendum to the Final Exam}
  You will also be able to make two attempts in the final exam period. This is in addition to the final exam. You will have your choice of any two essential questions, even two of the same. This is intended to be your last chance to demonstrate proficiency, not the main chance.

\fi

\ifonline

\else
\subsection{Extra Credit}\label{sec:ExtraCredit}
Extra credit, equivalent to two pre quiz grades, is available for the
course.  

This term the extra credit will focus on providing detailed worked examples of the practice questions.  Each practice question has a linked page in the wiki.  The link is given as feedback after you have taken the practice quiz.  There are more than 70 questions that need answers.  Questions needing detailed answers are tagged with ``answerneeded'' in the wiki.

To earn extra credit:
\begin{enumerate}
    
    \item Email me your name, a link to the page you will edit, and your wikidot username within one week before or after the due date of the associated PRE quiz. 
    
    \item Edit the wiki page and construct your answer within the same time frame, a week before or after the due date of the associated PRE quiz.
    
    \item Encourage your fellow students, ideally through piazza, to look at your answer, perhaps improve it, discuss it using the discuss tab, and finally give it an up or down vote using the voting tab on each page.  
    
    \item I will evaluate the worked answer if within two weeks of your email the net rating of the response is at least +5 when I make my daily check, and give /emph{one pre quiz worth of extra credit} to one of the writers if warranted.

\end{enumerate}

Please note that it is possible for a student to write a poor answer that never never gains enough positive responses for me to evaluate.  It is also possible for a poorly written answer to be edited and improved sufficiently for the extra credit to be granted to someone other than the original author.
\fi


\section{Other Rules}
\begin{itemize}


\item Begging for grades will result in an immediate lowering of your
  course grade by a full letter grade.
  
  \item When completing online quizzes you may use your book, wiki, calculator, spreadsheets, notes, or other resources as long as it is not another student or person.  The work must be authentically and genuinly your own. In other words, if you are copying answers you found online, it is not your work.
  
\ifonline

\else
   \item When completing essential skills tests, also known as essential questions, you may only use your calculator.  Your notes, wiki, book and other resources are not allowed.

\fi
\item Go to office hours at the first sign of trouble -- not as a last
  resort.

\item In this classroom, we support and value diversity.  To do so requires that we:
\begin{itemize}
   \item Respect the dignity and essential worth of all individuals
   \item Promote a culture of respect toward all individuals
    \item Respect the privacy, property, and freedom of others
    \item Reject bigotry, discrimination, violence, or intimidation of any kind
    \item Practice personal and academic integrity and expect it from others
   \item Promote the diversity of opinions, ideas, and backgrounds, which is
    the lifeblood of a university
\end{itemize}

 For additional information, please see the Office of Affirmative Action \& Equal Opportunity at \url{http://www.pdx.edu/diversity/affirmative-action}.


\item Accommodations are collaborative efforts between students, faculty, and the Disability Resource Center.  If you have a documented disability and require accommodation, you must arrange to meet with the course instructor prior to or within the first week of the term.  The documentation of your disability must come in writing from the Disability Resource Center (Faculty letter).  Students who believe they are eligible for accommodations but who have not yet obtained approval through the DRC should contact the DRC immediately.  Reasonable and appropriate accommodations will be provided for students with documented disabilities.  For more information on the Disability Resource Center, please see \url{http://www.drc.pdx.edu/}. 

\item Academic honesty is expected and required of students enrolled
  in this course.  Suspected academic dishonesty in this course will
  be handled according to the procedures set out in the Student Code
  of Conduct.

\item I am sympathetic to family emergencies but you must inform me as
  soon as possible. If the notice is verbal, please email me with your
  understanding of our agreement. All agreements have to be in
  writing.


\end{itemize}


\section{Expected Outline}
A link to a full, detailed, and continuously updated schedule will be
posted on D2L and the course wiki.


\section{Changes From Previous Terms}
The format and rules for the class change from term to term and are often the result of a discussion about how to improve the course conducted at the end of the term. The general changes from the Spring 2015 are:

\begin{itemize}
    \item Practice quizzes now have a link to a wiki page on that specific question.
     \item In-class quizzes, which were essentially participation grades, have been eliminated.
     \item Extra credit is now available by providing detailed answers to practice questions in a timely manner that are demonstrated to be useful to students.
    \item The wiki now has a voting system to flag both good and bad pages.
     \item The essential questions now work on a sliding scale with only two required for a passing grade and all four required for an A.
    \item The comprehensive written final exam is now composed of questions not in the POST quizzes .
     \item A more FERPA compliant method to return essential questions was added.
    \item Pre quizzes now show which questions were correct and not just the total score.
     \item Mandatory 24hr spacing of Pre quiz attempts has been removed.
    \item Pre quizzes make use of randomized question sets and not just parameters and question order.
    \item There is now a final exam.
    \item Post quizzes now have more open response questions that are a step closer to what will be on the final exam.
\end{itemize}

\end{document}
